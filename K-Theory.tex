\documentclass[a4paper,10pt]{article}
\usepackage[utf8x]{inputenc}

\usepackage{amsthm}
\usepackage{amsmath}
\usepackage{amssymb}

%Set up environments
\theoremstyle{plain}% default
\newtheorem{thm}{Theorem}
\newtheorem{lem}[thm]{Lemma}
\newtheorem{prop}[thm]{Proposition}
\newtheorem{cor}{Corollary}
\theoremstyle{definition}
\newtheorem{defn}{Definition}
\newtheorem{conj}{Conjecture}
\newtheorem{exmp}{Example}[section]
\theoremstyle{remark}
\newtheorem{rem}{Remark}

%Useful macros
%Blackboard bold
\newcommand{\RR}{\mathbb{R}} 
\newcommand{\CC}{\mathbb{C}} 
\newcommand{\HH}{\mathbb{H}} 

%opening
\title{K - Theory}
\author{Based on lectures by Professor John Jones}

\begin{document}

\maketitle

\section{Vector Bundles}

\subsection{Basic Definitions}

K theory is principally concerned with the study of Vector bundles over differential manifolds. So we first cover the basics of vector bundles. 

\begin{defn}
 A real vector bundle over a topological space $B$ is a topological space $E$ and a map $\pi:E\mapsto B$ such that:
 \begin{itemize}
  \item The structure of a real vector space on the fibre $E_b:=\pi^{-1}(b)$
  \item The vector space must be locally trivial so for each $b\in B$ we have a neighbourhood $U\subset B$ 
with $b\in U$ and a homeomorphism $h:U\times \mathbb{R}^n\mapsto E_U := \pi^{-1}(U)$ such that the following holds:
  \begin{itemize}
   \item The following diagram commutes:
% Dia 1
   \item For all $x\in U$ the natural map $v\mapsto h(x,v)$ is a linear isomorphism.
  \end{itemize}
 \end{itemize}
\end{defn}

The space $E$ is traditionally referred to as the total space, $B$ the base space and $\pi$ the projection.

\begin{defn}
 A map $\phi E\mapsto E'$ is a bundle morphism if it takes fibres linearly to fibres thus if:
  \begin{itemize}
   \item The following diagram commutes
% Dia 2
   \item For each $b \in B$, $\phi_b:=\phi\rvert_b :E_b \mapsto E'_b$ is a linear map
  \end{itemize}
\end{defn}

As one would expect a subbundle of a vector bundle $\pi:E\mapsto B$ is a subspace $E'\subset E$
such that the map induced by the inclusion $i:E'\mapsto E$ is a bundle morphism. Two 
vector bundles $E$ and $E'$ are isomorphic if there is a homeomorphism between them which is a bundle map.
Particularly the linear map $\phi_b:=\phi\rvert_b :E_b \mapsto E'_b$ is an isomorphism of vector spaces.

\begin{defn}
  $Vect(B)$ is the set of isomorphism classes of vector bundles over $B$ 
\end{defn}

Later we will show that $Vect(B)$ has a some algebraic structure, it is a semi-ring.

\subsection{Examples}

\subsection{Constructions}

\section{K-Theory}


\end{document}
